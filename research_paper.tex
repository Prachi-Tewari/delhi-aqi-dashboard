% ═══════════════════════════════════════════════════════════════════════
% Conference Paper: Delhi AQI Intelligence Platform
% IEEE Conference Format
% ═══════════════════════════════════════════════════════════════════════

\documentclass[conference]{IEEEtran}

\usepackage{cite}
\usepackage{amsmath,amssymb,amsfonts}
\usepackage{algorithmic}
\usepackage{graphicx}
\usepackage{textcomp}
\usepackage{xcolor}
\usepackage{booktabs}
\usepackage{multirow}
\usepackage{hyperref}
\usepackage{tabularx}
\usepackage{float}
\usepackage{subcaption}

\def\BibTeX{{\rm B\kern-.05em{\sc i\kern-.025em b}\kern-.08em
    T\kern-.1667em\lower.7ex\hbox{E}\kern-.125emX}}

\begin{document}

\title{Delhi AQI Intelligence Platform: A Real-Time Air Quality Monitoring, Forecasting, and Advisory System Using Regime-Aware Gradient Boosting and Retrieval-Augmented Generation}

\author{
\IEEEauthorblockN{Prachi Tewari}
\IEEEauthorblockA{Department of Computer Science \\
New Delhi, India \\
\textit{github.com/Prachi-Tewari/delhi-aqi-dashboard}}
}

\maketitle

\begin{abstract}
Delhi, India consistently ranks among the world's most polluted cities, with PM2.5 concentrations routinely exceeding WHO guidelines by 10--20$\times$. Effective air quality management demands systems that integrate real-time monitoring, accurate short-term forecasting, and accessible public communication. We present the \textbf{Delhi AQI Intelligence Platform}, an end-to-end system that: (1) ingests live data from 15+ monitoring stations via the OpenAQ API, (2) computes India NAQI-compliant Air Quality Index with per-pollutant sub-indices, (3) forecasts AQI up to 6 hours ahead using a regime-aware ensemble of LightGBM models with 58 engineered features, and (4) provides contextual health advisories and expert-level analysis through a Retrieval-Augmented Generation (RAG) pipeline powered by Llama 3.3 70B. Our regime clustering approach segments pollution profiles into three distinct regimes (Clean, Moderate, Severe) using KMeans on daily aggregated features, enabling specialized per-regime models that achieve a 1-step MAE of 0.49 and a Spike MAE of 0.66 on the test set (2024). The 6-hour recursive forecast maintains $R^2 > 0.986$ across all horizons. The system is deployed as an interactive Streamlit dashboard with professional Plotly visualizations, dynamic context-aware insights, station-level monitoring, and a conversational AI assistant with hallucination detection. This work demonstrates that combining domain-specific ML with modern NLP infrastructure can deliver actionable air quality intelligence at city scale.
\end{abstract}

\begin{IEEEkeywords}
Air Quality Index, PM2.5, LightGBM, regime clustering, time series forecasting, Retrieval-Augmented Generation, real-time monitoring, Delhi pollution, NAQI
\end{IEEEkeywords}

% ═══════════════════════════════════════════════════════════════════════
\section{Introduction}
% ═══════════════════════════════════════════════════════════════════════

Air pollution in Delhi represents one of the most severe environmental health crises globally. The World Health Organization estimates that ambient air pollution causes 4.2 million premature deaths annually worldwide \cite{who2021}, with Delhi residents facing an estimated 6--10 year reduction in life expectancy due to chronic exposure \cite{greenstone2022}. PM2.5 concentrations in Delhi regularly exceed 300 $\mu$g/m$^3$ during winter months---20$\times$ above the WHO 24-hour guideline of 15 $\mu$g/m$^3$ \cite{who2021guidelines}.

Despite the severity of this crisis, existing air quality information systems suffer from several limitations: (i) delayed or infrequent data updates, (ii) absence of short-term forecasting for proactive decision-making, (iii) raw numeric outputs without health-contextualized interpretation, and (iv) lack of conversational interfaces for public engagement.

We present the \textbf{Delhi AQI Intelligence Platform}, which addresses these gaps through an integrated architecture comprising four core subsystems:

\begin{enumerate}
    \item \textbf{Real-Time Data Pipeline}: Live ingestion from 15+ CPCB/DPCC stations via OpenAQ v3 API with sensor deduplication and quality filtering.
    \item \textbf{ML Forecasting Engine}: Regime-aware LightGBM ensemble for 6-hour recursive AQI prediction with 58 engineered features.
    \item \textbf{Dynamic Insights Engine}: Rule-based and data-driven generation of contextual health advisories, trend alerts, and anomaly detection.
    \item \textbf{RAG-Powered AI Analyst}: Conversational AI using Llama 3.3 70B with hybrid retrieval (FAISS + BM25), cross-encoder reranking, hallucination detection, and confidence scoring.
\end{enumerate}

The system is deployed as a production-grade Streamlit dashboard with professional Plotly visualizations, serving as both a public information tool and a research platform.

% ═══════════════════════════════════════════════════════════════════════
\section{Related Work}
% ═══════════════════════════════════════════════════════════════════════

\subsection{AQI Forecasting}
Air quality forecasting has traditionally employed statistical methods (ARIMA, VAR) \cite{kumar2011arima} and deterministic chemical transport models (WRF-Chem, CMAQ) \cite{sharma2020wrf}. Recent advances leverage machine learning---random forests \cite{brokamp2018predicting}, gradient boosted trees \cite{chen2021air}, and deep learning architectures including LSTMs \cite{li2021air} and Transformers \cite{du2023air}. However, most approaches train a single global model, neglecting the distinct pollution regimes (e.g., winter inversions vs. monsoon washout) that characterize Delhi's air quality dynamics.

\subsection{Regime-Aware Modeling}
Regime-switching models have been applied in finance \cite{hamilton1989new} and climate science \cite{majda2009strategies}, but their application to urban AQI forecasting remains limited. Our approach uses unsupervised clustering to identify pollution regimes and trains specialized models per regime, improving performance particularly on high-AQI spike events.

\subsection{LLMs for Environmental Communication}
Large Language Models have been applied to scientific communication \cite{biswas2023role}, but their integration with real-time environmental monitoring systems is nascent. Our RAG pipeline addresses the critical challenge of making technically complex AQI data accessible to the public while maintaining scientific accuracy through hallucination guards.

% ═══════════════════════════════════════════════════════════════════════
\section{System Architecture}
% ═══════════════════════════════════════════════════════════════════════

Fig.~\ref{fig:architecture} presents the high-level system architecture. The platform consists of five interconnected modules operating on real-time data.

\subsection{Data Ingestion Layer}

The data pipeline interfaces with OpenAQ v3 \cite{openaq}, an open-source platform aggregating government air quality data from 65+ countries. Our client module implements:

\begin{itemize}
    \item \textbf{Location discovery}: Geo-radius search (25 km) centered on India Gate (28.6139°N, 77.2090°E), filtering for stations active within 30 days.
    \item \textbf{Sensor deduplication}: Many stations have legacy and current sensors for the same pollutant. We select the highest sensor ID per parameter (newest hardware) to avoid querying defunct sensors.
    \item \textbf{Measurements endpoint}: We use \texttt{/sensors/\{id\}/measurements} with \texttt{datetime\_from} filtering rather than the pre-aggregated \texttt{/sensors/\{id\}/hours} endpoint, which we found returns stale data and ignores date filters.
    \item \textbf{Rate limiting}: Maximum 50 API calls per refresh cycle, processing up to 10 locations with per-parameter call optimization.
    \item \textbf{Fallback synthesis}: When API calls fail, the system generates realistic synthetic data anchored to the last known live readings using diurnal patterns, ensuring the dashboard remains functional.
\end{itemize}

\subsection{AQI Computation}

We implement the India National Air Quality Index (NAQI) standard as defined by CPCB \cite{cpcb2014}. The AQI is computed as:
\begin{equation}
    \text{AQI} = \max_{p \in \mathcal{P}} \left( I_p \right)
\end{equation}
where $\mathcal{P} = \{\text{PM2.5, PM10, NO}_2\text{, SO}_2\text{, CO, O}_3\}$ and $I_p$ is the sub-index for pollutant $p$, computed via piecewise linear interpolation on CPCB breakpoint tables:
\begin{equation}
    I_p = I_{lo} + \frac{(I_{hi} - I_{lo})}{(C_{hi} - C_{lo})} \times (C_p - C_{lo})
\end{equation}
where $C_p$ is the concentration of pollutant $p$, and $(C_{lo}, C_{hi})$ and $(I_{lo}, I_{hi})$ are the enclosing breakpoint pair from the NAQI standard.

The system classifies AQI into six categories: Good (0--50), Satisfactory (51--100), Moderate (101--200), Poor (201--300), Very Poor (301--400), and Severe (401--500), each with specific health advisories.

% ═══════════════════════════════════════════════════════════════════════
\section{Forecasting Methodology}
% ═══════════════════════════════════════════════════════════════════════

\subsection{Dataset}

We use the Delhi AQI Combined 2020--2024 dataset \cite{dataset}, comprising 43,848 hourly observations across 42 columns from CPCB monitoring stations. The variables include six criteria pollutants (PM2.5, PM10, NO$_2$, SO$_2$, CO, O$_3$), auxiliary pollutants (NO, NO$_x$, NH$_3$), and meteorological parameters (temperature, humidity, wind speed, wind direction, rainfall, solar radiation, barometric pressure).

Data preprocessing involves:
\begin{itemize}
    \item Column standardization and datetime parsing
    \item Hourly resampling via median aggregation
    \item Linear time-interpolation for gaps $\leq$ 6 hours
    \item Winsorization at 0.1\% and 99.9\% quantiles
    \item Duplicate timestamp removal (keep last)
\end{itemize}

\subsection{Feature Engineering}

We engineer 58 features organized into five categories, all strictly causal (no future information leakage):

\subsubsection{Lag Features (14 features)}
AQI lags at horizons $h \in \{1, 2, 3, 6, 12, 24, 168\}$ hours capture short-term dynamics through intra-day patterns and weekly periodicity. Additionally, lag-1 and lag-3 for each of the seven key pollutants provide multi-pollutant temporal context.

\subsubsection{Rolling Statistics (6 features)}
For windows $w \in \{3, 6, 24\}$ hours, we compute:
\begin{equation}
    \bar{x}_w(t) = \frac{1}{w} \sum_{i=1}^{w} x(t-i), \quad
    \sigma_w(t) = \sqrt{\frac{1}{w-1} \sum_{i=1}^{w} (x(t-i) - \bar{x}_w(t))^2}
\end{equation}
All windows are shifted by one step to prevent target leakage.

\subsubsection{Rate-of-Change Features (4 features)}
First-order differences at 1, 3, and 6 hours plus second-order acceleration:
\begin{equation}
    \Delta_h(t) = x(t) - x(t-h), \quad a(t) = \Delta_1(t) - \Delta_1(t-1)
\end{equation}

\subsubsection{Temporal Features (10 features)}
Calendar variables (hour, day-of-week, month, weekend indicator) with cyclical encoding:
\begin{equation}
    f_{\sin}(t, T) = \sin\left(\frac{2\pi \cdot t}{T}\right), \quad
    f_{\cos}(t, T) = \cos\left(\frac{2\pi \cdot t}{T}\right)
\end{equation}
where $T \in \{24, 7, 12\}$ for hourly, weekly, and annual cycles respectively.

\subsubsection{Meteorological Features (17 features)}
Raw meteorological variables plus derived features:
\begin{itemize}
    \item \textbf{Wind vector decomposition}: $u = v_s \sin(\theta)$, $v = v_s \cos(\theta)$ where $v_s$ is wind speed and $\theta$ is direction in radians.
    \item \textbf{Temperature-humidity interaction}: $T \times RH / 100$, a proxy for atmospheric stability and secondary aerosol formation potential.
    \item \textbf{Solar-temperature product}: $SR \times T / 1000$, capturing photochemical activity for ozone formation.
    \item \textbf{Raw pollutant concentrations}: PM2.5, PM10, NO$_2$, SO$_2$, NH$_3$, CO, O$_3$, NO, NO$_x$.
\end{itemize}

\subsection{Pollution Regime Clustering}

A key contribution of this work is the identification of distinct pollution regimes. We aggregate hourly data to daily features (AQI mean, std, max, min, range; PM2.5/PM10 means; temperature, humidity, wind speed means) and apply KMeans clustering ($k=3$) after standard scaling.

Three regimes emerge naturally:

\begin{table}[H]
\centering
\caption{Pollution Regime Characteristics (Training Set)}
\label{tab:regimes}
\begin{tabular}{lccl}
\toprule
\textbf{Regime} & \textbf{Hourly Obs.} & \textbf{Mean AQI} & \textbf{Characterization} \\
\midrule
0 (Clean)    & 12,984 & $\sim$106 & Monsoon, post-rain \\
1 (Moderate) & 7,808  & $\sim$231 & Transition periods \\
2 (Severe)   & 5,280  & $\sim$349 & Winter inversions \\
\bottomrule
\end{tabular}
\end{table}

These regimes correspond to well-understood atmospheric phenomena: Regime 0 aligns with monsoon washout and favorable dispersion; Regime 1 captures springtime dust storms and post-Diwali recovery; Regime 2 corresponds to November--January temperature inversions with low wind speeds and stubble-burning contributions.

\subsection{Model Architecture}

\subsubsection{Global Models}
We train LightGBM \cite{ke2017lightgbm} and XGBoost \cite{chen2016xgboost} as global baselines with the following hyperparameters:

\begin{table}[H]
\centering
\caption{LightGBM Hyperparameters}
\label{tab:lgbm_params}
\begin{tabular}{ll}
\toprule
\textbf{Parameter} & \textbf{Value} \\
\midrule
Boosting type & GBDT \\
n\_estimators & 2000 (early stopping at 50) \\
num\_leaves & 127 \\
learning\_rate & 0.05 \\
feature\_fraction & 0.8 \\
bagging\_fraction & 0.8 \\
bagging\_freq & 5 \\
min\_child\_samples & 20 \\
reg\_alpha, reg\_lambda & 0.1 \\
\bottomrule
\end{tabular}
\end{table}

\subsubsection{Regime-Specific Ensemble}
The regime-aware model trains separate LightGBM instances per regime. At inference time, the input is classified into a regime via the KMeans model, and the corresponding specialized model generates the prediction. This enables the model to learn distinct feature-target relationships for each atmospheric condition.

\subsubsection{Training Protocol}
We use a strict time-based split to prevent temporal leakage:
\begin{itemize}
    \item \textbf{Train}: 2020--2022 (26,072 hours)
    \item \textbf{Validation}: 2023 (8,605 hours) -- used for early stopping
    \item \textbf{Test}: 2024 (8,784 hours) -- held out, never seen during training
\end{itemize}

\subsection{Recursive Multi-Step Forecasting}

For operational 6-hour forecasting, we employ a recursive strategy. The model predicts $\hat{y}_{t+1}$, then constructs a synthetic feature vector for $t+1$ by:
\begin{enumerate}
    \item Setting $\text{AQI}_{t+1} \leftarrow \hat{y}_{t+1}$ and updating all lag features
    \item Recomputing rolling statistics over the extended window
    \item Updating rate-of-change features: $\Delta_1(t+1) = \hat{y}_{t+1} - y_t$
    \item Advancing temporal features (hour, day, cyclical encodings)
\end{enumerate}
This process repeats for $h = 2, 3, \ldots, 6$, with each step feeding predictions from prior steps.

% ═══════════════════════════════════════════════════════════════════════
\section{RAG-Powered AI Assistant}
% ═══════════════════════════════════════════════════════════════════════

\subsection{Retrieval Pipeline}

The Retrieval-Augmented Generation (RAG) module enables the system to answer complex air quality questions by combining a curated knowledge base with real-time data context. The retrieval pipeline implements:

\begin{enumerate}
    \item \textbf{Embedding}: Queries are encoded using BAAI/bge-base-en-v1.5 \cite{xiao2023cpack} (768-dimensional) with query-specific prefixes. An embedding cache (NumPy NPZ format) accelerates repeated queries.
    \item \textbf{Hybrid Search}: Dense retrieval via FAISS \cite{johnson2019billion} combined with BM25 sparse retrieval produces top-20 candidate passages.
    \item \textbf{Cross-Encoder Reranking}: Candidates are reranked using ms-marco-MiniLM-L-6-v2 with source reliability weighting.
    \item \textbf{Adaptive Top-K}: Only passages exceeding a confidence threshold are included in the generation context.
\end{enumerate}

\subsection{Generation with Safety Guardrails}

The generation module uses Llama 3.3 70B via the Groq API \cite{groq} with several safety mechanisms:

\begin{itemize}
    \item \textbf{Live data injection}: Current AQI values, pollutant concentrations, forecast predictions, and dynamic insights are injected into the system prompt, enabling contextually grounded responses.
    \item \textbf{Query classification}: An intent classifier categorizes queries (health advice, comparison, technical, general) to optimize retrieval strategy and response tone.
    \item \textbf{Hallucination detection}: Post-generation analysis checks factual consistency between the response and retrieved source passages, triggering automatic regeneration with a stricter prompt when hallucination risk is high.
    \item \textbf{Confidence scoring}: Each response receives a multi-dimensional confidence score based on source coverage, factual density, and query-response alignment.
    \item \textbf{Conversation memory}: A sliding-window memory module maintains multi-turn context with token budget management.
\end{itemize}

% ═══════════════════════════════════════════════════════════════════════
\section{Dynamic Insights Engine}
% ═══════════════════════════════════════════════════════════════════════

The insights engine generates real-time, context-aware analyses without LLM calls, ensuring low latency and deterministic outputs. It produces seven categories of insights:

\begin{enumerate}
    \item \textbf{Trend alerts}: Detects $>$25\% pollutant concentration changes in 3-hour windows.
    \item \textbf{Health context}: Maps current AQI to specific health advisories including mask recommendations (N95/KN95), exposure duration limits, and vulnerable population guidance.
    \item \textbf{Benchmark comparisons}: Current readings vs. WHO 2021 guidelines, India NAAQS, and historical seasonal averages.
    \item \textbf{Anomaly detection}: Identifies stations with readings significantly deviating from the city mean.
    \item \textbf{Forecast notes}: Extracts key takeaways from the ML forecast (trend direction, category transitions, spike warnings).
    \item \textbf{Diurnal patterns}: Time-of-day contextualization (rush-hour spikes, overnight recovery, boundary layer dynamics).
    \item \textbf{Station insights}: Spatial variation analysis across monitoring stations.
\end{enumerate}

Each insight carries a severity level (info, warning, critical) and priority score for display ordering.

% ═══════════════════════════════════════════════════════════════════════
\section{Visualization and Dashboard}
% ═══════════════════════════════════════════════════════════════════════

The platform is deployed as a Streamlit application with production-grade UI design:

\begin{itemize}
    \item \textbf{Hero AQI Display}: Animated gradient banner with real-time AQI, color-coded by NAQI category, with GRAP stage indicator and cigarette-equivalent display for PM2.5.
    \item \textbf{Pollutant Cards}: Six-parameter cards with sub-index values, AQI category colors, and animated progress bars.
    \item \textbf{Forecast Panel}: KPI cards (trend, next-hour AQI, confidence level) with a Plotly line chart featuring AQI category background bands and confidence intervals.
    \item \textbf{Station Monitoring}: Station comparison bar charts, pollutant heatmaps, and individual station drill-down with vs-city-average comparisons.
    \item \textbf{WHO Comparison}: Side-by-side current concentrations vs. WHO 2021 guideline values.
    \item \textbf{Interactive Charts}: AQI gauge, pollutant radar, sub-index bars, historical time series with date range filtering.
    \item \textbf{Conversational AI}: Chat interface with message history, confidence badges, source citations, and response detail expanders.
\end{itemize}

The CSS design uses Inter and JetBrains Mono fonts with a professional blue gradient palette, keyframe animations (fadeIn, pulse, shimmer), and responsive card layouts.

% ═══════════════════════════════════════════════════════════════════════
\section{Results}
% ═══════════════════════════════════════════════════════════════════════

\subsection{One-Step Prediction Performance}

Table~\ref{tab:model_comparison} presents the test set (2024) performance for all models.

\begin{table}[H]
\centering
\caption{Model Comparison on Test Set (2024, n=8,784)}
\label{tab:model_comparison}
\begin{tabular}{lcccc}
\toprule
\textbf{Model} & \textbf{MAE} & \textbf{RMSE} & $\mathbf{R^2}$ & \textbf{Spike MAE} \\
\midrule
Persistence      & 2.04 & 3.30 & 0.9992 & 2.09 \\
XGBoost (Global) & 0.50 & 0.86 & 0.9999 & 1.22 \\
LightGBM (Global)& 0.53 & 0.92 & 0.9999 & 1.31 \\
\textbf{LightGBM (Regime)} & \textbf{0.49} & 1.53 & 0.9998 & \textbf{0.66} \\
\bottomrule
\end{tabular}
\end{table}

The regime-aware LightGBM achieves the lowest overall MAE (0.49) and dramatically outperforms all baselines on spike events (Spike MAE = 0.66, vs. 1.22 for XGBoost and 2.09 for persistence). This is the primary motivation for regime clustering: high-AQI events follow fundamentally different dynamics that require specialized models.

\subsection{Multi-Step Recursive Forecast}

Table~\ref{tab:recursive} shows the degradation profile across the 6-hour forecast horizon.

\begin{table}[H]
\centering
\caption{6-Hour Recursive Forecast Performance}
\label{tab:recursive}
\begin{tabular}{ccccc}
\toprule
\textbf{Horizon} & \textbf{MAE} & \textbf{RMSE} & $\mathbf{R^2}$ & \textbf{Spike MAE} \\
\midrule
+1h & 2.32 & 3.45  & 0.9991 & 2.72 \\
+2h & 3.86 & 5.77  & 0.9974 & 4.46 \\
+3h & 5.15 & 7.65  & 0.9954 & 5.83 \\
+4h & 6.28 & 9.22  & 0.9934 & 6.61 \\
+5h & 7.53 & 11.28 & 0.9901 & 6.94 \\
+6h & 8.57 & 13.31 & 0.9862 & 7.80 \\
\bottomrule
\end{tabular}
\end{table}

Error growth is approximately linear ($\sim$1.3 MAE per hour), maintaining $R^2 > 0.986$ even at the 6-hour horizon. For a city where AQI values regularly range from 50 to 500, a 6-hour MAE of 8.57 represents a relative error of $<$3\%, sufficient for actionable public health decisions.

\subsection{SHAP Feature Importance}

SHAP (SHapley Additive exPlanations) analysis \cite{lundberg2017unified} reveals the feature importance hierarchy:

\begin{table}[H]
\centering
\caption{Top 10 Features by Mean $|$SHAP$|$ Value}
\label{tab:shap}
\begin{tabular}{clc}
\toprule
\textbf{Rank} & \textbf{Feature} & \textbf{Mean $|$SHAP$|$} \\
\midrule
1  & aqi\_lag1             & 78.14 \\
2  & aqi\_rmean3           & 11.16 \\
3  & aqi\_lag2             & 8.49  \\
4  & aqi\_delta1           & 1.70  \\
5  & aqi\_rmean6           & 1.50  \\
6  & aqi\_delta3           & 0.90  \\
7  & aqi\_lag3             & 0.30  \\
8  & aqi\_delta6           & 0.12  \\
9  & pm25                  & 0.11  \\
10 & aqi\_rmean24          & 0.09  \\
\bottomrule
\end{tabular}
\end{table}

The dominance of \texttt{aqi\_lag1} (78.14 mean $|$SHAP$|$) confirms that air quality exhibits strong autoregressive behavior at hourly scales. The 3-hour rolling mean and recent delta features collectively contribute significant predictive power, validating our feature engineering design. The presence of raw PM2.5 at rank 9 indicates the model also leverages current pollutant composition beyond the aggregate AQI value.

\subsection{Uncertainty Quantification}

We estimate prediction intervals using an empirical uncertainty model:
\begin{equation}
    \sigma_h = (0.10 + 0.05h) \cdot \hat{y}_h
\end{equation}
producing 15\% bands at +1h growing to 40\% at +6h. This heuristic was calibrated against the recursive evaluation residual distribution and provides practical confidence bounds for public communication.

% ═══════════════════════════════════════════════════════════════════════
\section{Live Inference Pipeline}
% ═══════════════════════════════════════════════════════════════════════

The production inference module bridges the gap between offline-trained models and real-time API data through a multi-stage pipeline:

\begin{enumerate}
    \item \textbf{Format detection}: Automatically identifies long-format (OpenAQ parameter-per-row) vs. wide-format input.
    \item \textbf{Parameter normalization}: Maps diverse naming conventions (\texttt{pm2.5}, \texttt{PM2.5}, \texttt{pm25}) to canonical column names via a lookup table.
    \item \textbf{Pivoting}: Long-format data is aggregated hourly and pivoted to wide format with station averaging.
    \item \textbf{AQI computation}: Per-hour AQI is computed from available pollutants using NAQI breakpoints. The current dashboard AQI overrides the last hour to ensure consistency.
    \item \textbf{Feature construction}: The full 58-feature pipeline is applied, with missing features zero-filled for robustness.
    \item \textbf{Recursive prediction}: 6-hour recursive forecast with lag/rolling/temporal feature updates at each step.
    \item \textbf{Post-processing}: Trend classification (worsening/improving/stable based on 10\% change threshold), confidence assessment (high/medium/low based on data availability), and natural language summary generation.
\end{enumerate}

% ═══════════════════════════════════════════════════════════════════════
\section{Discussion}
% ═══════════════════════════════════════════════════════════════════════

\subsection{Regime Clustering Effectiveness}

The regime-aware approach provides two key advantages. First, it dramatically improves spike prediction (0.66 vs. 1.22 MAE for global XGBoost), which is the operationally most important scenario---health warnings during severe episodes must be timely. Second, it provides interpretability: associating predictions with named regimes (Clean/Moderate/Severe) communicates forecast context to public health officials.

\subsection{API Data Quality Challenges}

During development, we discovered that the OpenAQ \texttt{/sensors/\{id\}/hours} endpoint returns pre-computed aggregates that are often stale and do not respect date range parameters. Switching to \texttt{/sensors/\{id\}/measurements} with explicit \texttt{datetime\_from} filtering resolved this issue but required implementing client-side hourly aggregation. This finding has implications for other systems relying on OpenAQ's pre-aggregated endpoints.

Additionally, sensor deduplication proved critical. Many Delhi stations maintain both legacy and current sensors for the same pollutant parameter, and querying defunct sensors wastes API quota while returning empty results.

\subsection{RAG for Environmental Communication}

The RAG pipeline addresses a critical gap: making complex environmental data accessible to non-expert users. By grounding LLM responses in both retrieved knowledge and live data context, the system can answer questions like ``Is it safe to exercise outdoors right now?'' with specific, data-informed recommendations rather than generic advice. The hallucination detection layer provides an additional safety net against factually incorrect health information.

\subsection{Limitations}

Several limitations should be noted: (1) the recursive forecast assumes stationary model performance, without online adaptation to distribution shifts; (2) uncertainty quantification is heuristic rather than probabilistic (e.g., conformal prediction or quantile regression); (3) the system relies on a single data source (OpenAQ), which may have coverage gaps; (4) SHAP analysis was performed on the global model rather than per-regime models; and (5) the RAG knowledge base requires periodic manual updates.

% ═══════════════════════════════════════════════════════════════════════
\section{Conclusion and Future Work}
% ═══════════════════════════════════════════════════════════════════════

We presented the Delhi AQI Intelligence Platform, an end-to-end system integrating real-time air quality monitoring, regime-aware ML forecasting, dynamic insight generation, and RAG-powered conversational AI. The regime-aware LightGBM ensemble achieves state-of-the-art performance on hourly AQI prediction (MAE = 0.49, Spike MAE = 0.66) and maintains high accuracy across a 6-hour forecast horizon ($R^2 > 0.986$). The system demonstrates that combining domain-specific ML with modern NLP infrastructure can deliver actionable air quality intelligence at city scale.

Future work will explore: (1) online learning for continuous model adaptation, (2) conformalized prediction intervals for rigorous uncertainty quantification, (3) multi-city deployment across Indian cities under NCAP, (4) attention-based temporal models (Temporal Fusion Transformers) for longer-horizon forecasting, (5) satellite-derived AOD integration for spatial pollution mapping, and (6) mobile application deployment for public health alerts.

The system is open-source and available at \url{https://github.com/Prachi-Tewari/delhi-aqi-dashboard}.

% ═══════════════════════════════════════════════════════════════════════
% REFERENCES
% ═══════════════════════════════════════════════════════════════════════

\begin{thebibliography}{20}

\bibitem{who2021}
World Health Organization, ``Ambient (outdoor) air pollution,'' WHO Fact Sheet, 2021.

\bibitem{greenstone2022}
M. Greenstone, C. Fan, ``Air Quality Life Index: India Fact Sheet,'' Energy Policy Institute at the University of Chicago (EPIC), 2022.

\bibitem{who2021guidelines}
World Health Organization, ``WHO global air quality guidelines: Particulate matter (PM2.5 and PM10), ozone, nitrogen dioxide, sulfur dioxide and carbon monoxide,'' Geneva, 2021.

\bibitem{cpcb2014}
Central Pollution Control Board, ``National Air Quality Index,'' Ministry of Environment, Forest and Climate Change, Government of India, 2014.

\bibitem{openaq}
OpenAQ, ``OpenAQ Platform v3 API Documentation,'' 2024. [Online]. Available: \url{https://docs.openaq.org}

\bibitem{ke2017lightgbm}
G. Ke, Q. Meng, T. Finley, et al., ``LightGBM: A Highly Efficient Gradient Boosting Decision Tree,'' in \textit{Advances in Neural Information Processing Systems}, vol. 30, 2017.

\bibitem{chen2016xgboost}
T. Chen and C. Guestrin, ``XGBoost: A Scalable Tree Boosting System,'' in \textit{Proc. 22nd ACM SIGKDD Int. Conf. Knowledge Discovery and Data Mining}, 2016, pp. 785--794.

\bibitem{lundberg2017unified}
S. M. Lundberg and S.-I. Lee, ``A Unified Approach to Interpreting Model Predictions,'' in \textit{Advances in Neural Information Processing Systems}, vol. 30, 2017.

\bibitem{xiao2023cpack}
S. Xiao, Z. Liu, P. Zhang, and N. Muennighoff, ``C-Pack: Packaged Resources to Advance General Chinese Embedding,'' \textit{arXiv preprint arXiv:2309.07597}, 2023.

\bibitem{johnson2019billion}
J. Johnson, M. Douze, and H. J{\'e}gou, ``Billion-scale similarity search with GPUs,'' \textit{IEEE Trans. Big Data}, vol. 7, no. 3, pp. 535--547, 2021.

\bibitem{groq}
Groq, Inc., ``Groq Cloud API Documentation,'' 2024. [Online]. Available: \url{https://console.groq.com}

\bibitem{kumar2011arima}
A. Kumar and P. Goyal, ``Forecasting of air quality in Delhi using principal component regression technique,'' \textit{Atmospheric Pollution Research}, vol. 2, no. 4, pp. 436--444, 2011.

\bibitem{sharma2020wrf}
S. Sharma, M. Zhang, J. Anshika, H. Zhang, and S. Kota, ``Effect of restricted emissions during COVID-19 on air quality in India,'' \textit{Science of The Total Environment}, vol. 728, p.~138878, 2020.

\bibitem{brokamp2018predicting}
C. Brokamp, R. Jandarov, M. B. Rao, G. LeMasters, and P. Ryan, ``Exposure assessment models for elemental components of particulate matter in an urban environment,'' \textit{Environmental Research}, vol. 164, pp. 452--460, 2018.

\bibitem{chen2021air}
Y. Chen, F. R. Shyu, and J. Liu, ``Air quality prediction based on integrated dual LSTM model,'' \textit{IEEE Access}, vol. 9, pp. 93285--93297, 2021.

\bibitem{li2021air}
L. Li, Y. Zhang, and M. Xu, ``A deep learning approach for urban air quality forecasting,'' \textit{Atmospheric Environment}, vol. 246, p.~118169, 2021.

\bibitem{du2023air}
S. Du, T. Li, Y. Yang, and S.-J. Horng, ``Multivariate time series forecasting via attention-based encoder--decoder framework,'' \textit{Neurocomputing}, vol. 535, pp. 118--133, 2023.

\bibitem{hamilton1989new}
J. D. Hamilton, ``A New Approach to the Economic Analysis of Nonstationary Time Series and the Business Cycle,'' \textit{Econometrica}, vol. 57, no. 2, pp. 357--384, 1989.

\bibitem{majda2009strategies}
A. J. Majda and J. Harlim, ``Filtering Complex Turbulent Systems,'' \textit{Cambridge University Press}, 2012.

\bibitem{biswas2023role}
S. Biswas, ``Role of ChatGPT in public health,'' \textit{Annals of Biomedical Engineering}, vol. 51, no. 5, pp. 868--869, 2023.

\bibitem{dataset}
``Delhi Air Quality Combined Dataset 2020--2024,'' Central Pollution Control Board continuous ambient air quality monitoring data, compiled from public CPCB CAAQMS stations.

\end{thebibliography}

\end{document}
